%
%==> Section: Conclusion
%
\section{
  Conclusion
}

%
%==> Conclusion
%
\begin{frame}
  \frametitle{
    Conclusion
  }

  \begin{enumerate}
  \item
    Ti$k$Z is an enormous program. You will find commands to draw hierarchical trees, to draw lots of different types of shapes, to do some elementary programming, to align elements of a picture in a matrix frame, to decorate nodes, to compute the intersections of paths, etc
  \item
     For a gentle introduction see: \href{http://cremeronline.com/LaTeX/minimaltikz.pdf}{A very minimal introduction to TikZ}.
  \item
    If you conundrum is not addressed in these slides it's probably somewhere in the exhaustive manual available at \href{http://paws.wcu.edu/tsfoguel/tikzpgfmanual.pdf}{tikzpgfmanual}
  \item
    Post questions on \href{http://tex.stackexchange.com/}{TeX - \LaTeX\ Stack Exchange}, a question and answer site for users of TeX, \LaTeX, ConTeXt, and related typesetting systems. It's 100\% free, no registration required. 
  \end{enumerate}
  
\end{frame}
